\documentclass[a4paper,11pt]{article}
\usepackage[T1]{fontenc}
\usepackage[utf8]{inputenc}
\usepackage{lmodern}

\title{MAC0431 - Segundo Semestre de 2012 \\
       EP2 - Fase 1: Proposta}
\author{Wilson Kazuo Mizutani \\ Thiago de Gouveia Nunes}

\begin{document}

\maketitle

\section{Introdução}

  Nesse trabalho tentaremos usar o Hadoop para extrair estatísticas de
  desempenho de combate no jogo \textit{World of Warcraft}. É característico
  desse jogo a constante busca dos jogadores pela melhor formulação possível de
  seus personagens virtuais de maneira a otimizar sua eficácia em combate.
  
  Um dos principais indicadores dessa eficácia é o DPS (Dano por Segundo ou
  \textit{Damage per Second}). No jogo, dano é quantificado numericamente, sendo
  possível, portanto, calcular o DPS dividindo os pontos de dano que um
  determinado personagem causou pelo intervalo de tempo que ele permaneceu ativo
  em combate.
  
  Além disso, o próprio jogo oferece a opção de gerar automaticamente relátorios
  dos combates nos quais o jogador participa. Eles contêm diversas informações,
  inclusive o dano causado pelos personagens que participaram do combate.
  Nossa proposta é de usar esses relatórios de combate como base da dados para
  extração do DPS médio de vários personagens.

\section{O Problema}

  Um exemplo de relatório de combate gerado pelo jogo \textit{World of Warcraft}
  pode ser visto no arquivo \textbf{\LARGE{TODO}}. Parte do trabalho envolverá
  um estudo sobre como os danos de combate são expressos nesses relatórios, mas
  sabe-se que as entradas de dano contêm os seguintes dados:
  
  \begin{itemize}
  
    \item Horário em que o dano foi causado.
    \item Que personagem causou o dano.
    \item Que personagem recebeu o dano.
    \item Quantos pontos de dano foram causados.
  
  \end{itemize}
  
  Dados vários relatórios gerados pelo jogo (ou seja, relatos de vários combates
  com diversos personagens) queremos gerar uma lista de itens contendo as
  seguintes informações de cada personagem citado nos relatórios:
  
  \begin{itemize}
  
    \item Nome do personagem.
    \item DPS médio do personagem.
  
  \end{itemize}
  
  Os argumentos que o programa receberá serão simplesmente a pasta que contém os
  relatórios a serem usados como entrada, e a pasta na qual deverá ser escrito o
  arquivo de saída contendo a listagem do DPS médio dos personagens.

\section{Abordagem}

  A ideia básica da nossa solução para o problema é encontrar nos relatórios do
  jogo os intervalos de tempo nos quais cada personagem permaneceu ativo em
  combate e somar a quantidade de dano que ele causou nesse intervalo. Com isso,
  teremos uma medida do DPS dele para uma certa variação de tempo. Juntando
  todas as medidas desse tipo que podem ser extraídas da base de dados,
  calculamos o DPS médio do personagem.
  
  Como é de se imaginar, a parte que envolve extrair o DPS dos personagens
  associando-o a intervalos de tempo corresponde a função \textbf{Map} desejada;
  enquanto que a parte de juntar essas medidas para compor o DPS médio
  (ponderado pelo tamanho dos intervalos de tempo de cada medida) corresponde à
  função \textbf{Reduce}. A função \textbf{Shuffle} apenas juntará medidas,
  somando a variação de tempo total e combinando o DPS usando como peso as
  variações de tempo individuais.
  
  Segue uma descrição mais precisa de cada uma dessas etapas.
  
  \subsection{Função \textbf{Map}}
  
  \subsection{Função \textbf{Shuffle}}
  
  \subsection{Função \textbf{Reduce}}

\end{document}
