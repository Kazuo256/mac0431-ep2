\documentclass[a4paper,11pt]{article}
\usepackage[T1]{fontenc}
\usepackage[utf8]{inputenc}
\usepackage{lmodern}

\title{MAC0431 - Segundo Semestre de 2012 \\
       EP2 - Fase 1: Proposta}
\author{Thiago de Gouveia Nunes - Nº USP: 6797289 \\
        Wilson Kazuo Mizutani - Nº USP: 6797230}

\begin{document}

\maketitle

\section{Introdução}

  Nesse trabalho tentaremos usar o Hadoop para extrair estatísticas de
  desempenho de combate no jogo \textit{World of Warcraft}. É característico
  desse jogo a constante busca dos jogadores pela melhor formulação possível de
  seus personagens virtuais de maneira a otimizar sua eficácia em combate.
  
  Um dos principais indicadores dessa eficácia é o DPS (Dano por Segundo ou
  \textit{Damage per Second}). No jogo, dano é quantificado numericamente, sendo
  possível, portanto, calcular o DPS dividindo os pontos de dano que um
  determinado personagem causou pelo intervalo de tempo que ele permaneceu ativo
  em combate.
  
  Além disso, o próprio jogo oferece a opção de gerar automaticamente relátorios
  dos combates nos quais o jogador participa. Eles contêm diversas informações,
  inclusive o dano causado pelos personagens que participaram do combate.
  Nossa proposta é de usar esses relatórios de combate como base da dados para
  extração do DPS médio de vários personagens.

\section{O Problema}

  Um exemplo de relatório de combate gerado pelo jogo \textit{World of Warcraft}
  pode ser visto no arquivo \textbf{\LARGE{wowlog}}. Parte do trabalho
  envolverá um estudo sobre como os danos de combate são expressos nesses
  relatórios, mas sabe-se que as entradas de dano contêm os seguintes dados:
  
  \begin{itemize}
  
    \item Horário em que o dano foi causado.
    \item Que personagem causou o dano.
    \item Que personagem recebeu o dano.
    \item Quantos pontos de dano foram causados.
  
  \end{itemize}
  
  Dados vários relatórios gerados pelo jogo (ou seja, relatos de vários combates
  com diversos personagens) queremos gerar uma lista de itens contendo as
  seguintes informações de cada personagem citado nos relatórios:
  
  \begin{itemize}
  
    \item Nome do personagem.
    \item DPS médio do personagem.
  
  \end{itemize}
  
  Os argumentos que o programa receberá serão simplesmente a pasta que contém os
  relatórios a serem usados como entrada, e a pasta na qual deverá ser escrito o
  arquivo de saída contendo a listagem do DPS médio dos personagens.

\section{Abordagem}

  A ideia básica da nossa solução para o problema é encontrar nos relatórios do
  jogo os intervalos de tempo nos quais cada personagem permaneceu ativo em
  combate e somar a quantidade de dano que ele causou nesse intervalo. Com isso,
  teremos uma medida do DPS dele para uma certa variação de tempo. Juntando
  todas as medidas desse tipo que podem ser extraídas da base de dados,
  calculamos o DPS médio do personagem.
  
  Como é de se imaginar, a parte que envolve extrair o DPS dos personagens
  associando-o a intervalos de tempo corresponde a função \textbf{Map} desejada;
  enquanto que a parte de juntar essas medidas para compor o DPS médio
  (ponderado pelo tamanho dos intervalos de tempo de cada medida) corresponde à
  função \textbf{Reduce}. A função \textbf{Shuffle} faz praticamente a mesma
  coisa que a \textbf{Reduce}, exceto que ao invés de gerar só o DPS médio ela
  também fornece a variação de tempo total usada para calcular essa média.
  
  Segue uma descrição mais precisa de cada uma dessas etapas.
  
  \subsection{Função \textbf{Map}}
  
    Assim como no exemplo do WordCount, os pares chave-valor de entrada para
    nossa função \textbf{Map} serão do tipo \verb$(Arquivo, Texto)$. Já os
    pares de saída serão do tipo \verb$(Nome, Medida)$, onde \verb$Medida$ será
    um valor de DPS associado a um intervalo de tempo (ou seja, dois números).
    
    Pseudo-código:
    
    \begin{verbatim}
    +-------------------------------------------------------+
    |  DPS-Map (Arquivo, Texto, Contexto):                  |
    |    P <- {}; // personagens que causaram dano          |
    |    D <- {}; // dano total de cada personagem          |
    |    T <- {}; // horário do primeiro dano do personagem |
    |    para cada linha em Texto:                          |
    |      se linha for uma entrada de dano:                |
    |        p <- nome do personagem que causou o dano;     |
    |        d <- dano causado;                             |
    |        t <- horário do dano;                          |
    |        se p não está em P:                            |
    |          P <- P u {p};                                |
    |          D[p] <- d;                                   |
    |          T[p] <- t;                                   |
    |        caso contrário:                                |
    |          D[p] <- D[p] + d;                            |
    |    f <- horário da última linha do Texto;             |
    |    para cada p em P:                                  |
    |      dt <- f - T[p];                                  |
    |      Medida <- (D[p]/dt, dt);                         |
    |      passar o par (p, Medida) para o Contexto;        |
    +-------------------------------------------------------+
    \end{verbatim}
  
  \subsection{Função \textbf{Reduce}}
  
    A função \textbf{Reduce} reduz várias medidas de um mesmo personagem para a
    média total do FPS dele. Cada medida de FPS é ponderada pela variação de
    tempo associada a ela. Assim, os pares chave-valor de entrada para ela serão
    do tipo \verb$(Nome, Medidas)$ - onde \verb$Medidas$ é uma lista de medidas
    -, e os de saída serão do tipo \verb$(Nome, DPS)$ - onde \verb$DPS$ é na
    verdade a média total do DPS do personagem.
    
    Pseudo-código:
    
    \begin{verbatim}
    +-------------------------------------------------------+
    |  DPS-Reduce (Nome, Medidas, Contexto):                |
    |    D <- 0; // DPS total do personagem                 |
    |    T <- 0; // tempo de combate total do personagem    |
    |    para cada Medida em Medidas:                       |
    |      // Sendo que:                                    |
    |      //   Medida[0] --> DPS                           |
    |      //   Medida[1] --> dt                            |
    |      D <- D + Medida[0]*Medida[1];                    |
    |      T <- T + Medida[1];                              |
    |    passar o par (Nome, D/T) para o Contexto;          |
    +-------------------------------------------------------+
    \end{verbatim}
    
  
  \subsection{Função \textbf{Shuffle}}
  
    A função \textbf{Shuffle} reduz várias medidas de um mesmo personagem em uma
    só. O cálculo é o mesmo usado na função \textbf{Reduce}, mudando apenas o
    formato da saída, que serão pares chave-valor do tipo \verb$(Nome, Medida)$,
    como na função \textbf{Map}.
    
    Pseudo-código:
    
    \begin{verbatim}
    +-------------------------------------------------------+
    |  DPS-Reduce (Nome, Medidas, Contexto):                |
    |    D <- 0; // DPS total do personagem                 |
    |    T <- 0; // tempo de combate total do personagem    |
    |    para cada Medida em Medidas:                       |
    |      D <- D + Medida[0]*Medida[1];                    |
    |      T <- T + Medida[1];                              |
    |    Medida <- (D/T, T);                                |
    |    passar o par (Nome, Medida) para o Contexto;       |
    +-------------------------------------------------------+
    \end{verbatim}

\end{document}