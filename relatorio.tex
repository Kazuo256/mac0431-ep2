\documentclass[a4paper,11pt]{article}
\usepackage[T1]{fontenc}
\usepackage[utf8]{inputenc}
\usepackage{lmodern}
\usepackage{hyperref}

\title{MAC0431 - Segundo Semestre de 2012 \\
       EP2 - Fase 2: Relatório}
\author{Thiago de Gouveia Nunes - Nº USP: 6797289 \\
        Wilson Kazuo Mizutani - Nº USP: 6797230}

\begin{document}

\maketitle

\section{Organização do Projeto}

  O diretório compactado entregue deve conter os seguintes arquivos:
  
  \begin{itemize}
    \item[\textbf{dpsmeter.jar}]
      Nosso projeto. É possível rodar ele no Hadoop e também importar como um
      projeto de Eclipse para ver o código fonte.
    \item[\textbf{proposta.pdf}]
      A proposta original do projeto, entregue na primeira fase.
    \item[\textbf{relatorio.pdf}]
      Esse relatório.
    \item[\textbf{slides.pdf}]
      Os slides da proposta original do projeto, entregues na primeira fase.
    \item[\textbf{wowlog.txt}]
      Um relatório pequeno de batalha gerado pelo jogo World of Warcraft para
      ser usado como entrada para o programa.
  \end{itemize}
  
\section{Modo de Uso}

  Para usar nosso programa, é preciso ter o Hadoop instalado e devidamente
  configurado, e então usar o comando
  
  \begin{verbatim} $ hadoop jar dpsmeter.jar <entrada> <saída> \end{verbatim}
  
  Onde:
  
  \begin{itemize}
    \item[<entrada>]
      Deve ser um arquivo de relatório de combate gerado pelo jogo World of
      Warcraft (versão atual) como o arquivo \textbf{wowlog.txt}.
    \item[<saída>]
      É o nome do diretório onde se deseja que a saída do programa seja escrita.
      Esse diretório é criado pelo Hadoop, portanto não deve existir quando o
      programa for executado.
  \end{itemize}
  
  Temos mais três relatórios de combate disponíveis
  \href{http://www.linux.ime.usp.br/~gorobaum/logs/}{nessa página}. Eles eram
  grandes demais para mandarmos junto com o projeto.
  
\section{Detalhes de Implementação}

  Em geral seguimos a ideia apresentada na proposta original no uso das funções
  Map e Reduce do Hadoop. A única grande diferença é que as estruturas "Medida",
  ao invés de guardarem o DPS (dano por segundo) de um personagem, guardam a
  quantidade de dano total que ele causou. Isso simplifica as contas na hora de
  juntar várias medidas nas funções Reduce e Shuffle.
  
  Para maiores detalhes de como era a ideia da proposta original, basta ler o
  arquivo \textbf{proposta.pdf} que entregamos junto com esse relatório.
  Resumidamente, temos que:
  
  \begin{enumerate}
    \item
      \textbf{Classe Map}:
      \begin{itemize}
        \item[\textit{Recebe:}] \verb$par(arquivo, texto)$
        \item[\textit{Devolve:}] \verb$par(nome, medida)$
        \item[\textit{Descrição:}]
          Lê o relatório de combate linha por linha, detectando as entradas de
          dano. Usa as informações de quem causou o dano e de quando o dano foi
          causado para gerar as medidas.
      \end{itemize}
    \item
      \textbf{Classe Shuffle}:
      \begin{itemize}
        \item[\textit{Recebe:}] \verb$par(nome, lista(medidas))$
        \item[\textit{Devolve:}] \verb$par(nome, medida)$
        \item[\textit{Descrição:}]
          Junta várias medidas de um mesmo personagens numa só. Como agora as
          medidas guardam o dano total e o tempo total associado ao dano
          causado, basta somar essas informações respectivamente entre medidas.
      \end{itemize}
    \item
      \textbf{Classe Reduce}:
      \begin{itemize}
        \item[\textit{Recebe:}] \verb$par(nome, lista(medidas))$
        \item[\textit{Devolve:}] \verb$par(nome, DPS)$
        \item[\textit{Descrição:}]
          Junta várias medidas como a classe Shuffle, mas no final calcula o
          DPS de cada personagem e passa para a saída final do programa.
      \end{itemize}
  \end{enumerate}
  
  Para informações de como os relatórios do World of Warcraft são estruturados,
  recomendamos \href{http://www.wowwiki.com/API_COMBAT_LOG_EVENT}{essa página da
  Web}, que apesar de desatualizada, descreve bem a ideia básica.
  
  Com o que aprendemos nesse link e observando os relatórios gerados, fizemos a
  classe Parser, usada na classe Map, que interpreta as linhas do relatório e
  detecta as entradas de dano. A classe Map passa essas entradas de dano para
  a classe DamageMeasurer, que mantém os dados de quanto dano cada personagem
  causou e quanto tempo cada personagem esteve ativo no combate. Com isso, ela
  gera objetos do tipo Measure, que são as nossas "medidas". Objetos do tipo
  Measure podem combinar-se, e isso é usado pelas classes Shuffle e Reduce para
  juntar as medidas e produzir o resultado final.
  
\section{Dificuldades e Facilidades}

  As duas maiores dificuldades que tivemos foram depurar classe Parser
  (principalmente a parte de ler e manipular unidades de tempo usando a classe
  Calendar do Java) e perceber que o Hadoop chama o método \verb$map$ de um
  mesmo objeto Map para cada linha do arquivo, o que exigiu cuidado na hora de
  escolher usar variáveis locais ou atributos de classe.
  
  Uma outra dificuldade foi que quando se instala o Hadoop usando um pacote DEB,
  a configuração padrão dele limita a memória usada para 128 MB. Depois de
  pesquisar na Internet, descobrimos que essa configuração pode ser editada no
  arquivo \textbf{hadoop-env.sh}. Mas existem dois arquivos desse: um na pasta
  onde o Hadoop é instalado, e outro em \verb$/etc/hadoop/$. Aquele que de fato
  deve ser editado é o segundo...
  
  As principais facilidades foram montar o ambiente de desenvolvimento (usando
  Eclipse e Git) e a programação em si do projeto.
  
  Ou seja, as dificuldades estavam mais nas partes inicial (instalação e
  configuração do Hadoop) e final (depuração em geral) do desenvolvimento.

\end{document}
