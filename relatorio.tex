\documentclass[a4paper,11pt]{article}
\usepackage[T1]{fontenc}
\usepackage[utf8]{inputenc}
\usepackage{lmodern}
\usepackage{hyperref}

\title{MAC0431 - Segundo Semestre de 2012 \\
       EP2 - Fase 2: Relatório}
\author{Thiago de Gouveia Nunes - Nº USP: 6797289 \\
        Wilson Kazuo Mizutani - Nº USP: 6797230}

\begin{document}

\maketitle

\section{Organização do Projeto}

  Nosso projeto é composto pelos seguintes arquivos:
  
  \begin{itemize}
    \item[\textbf{dpsmeter.jar}]
      O programa. É possível importar ele para o Eclipse.
    \item[\textbf{proposta.pdf}]
      A proposta original do projeto, entregue na primeira fase.
    \item[\textbf{relatorio.pdf}]
      Esse relatório.
    \item[\textbf{wowlog.txt}]
      Um relatório de batalha pequeno gerado pelo jogo World of Warcraft para
      ser usado como entrada para o programa.
  \end{itemize}

  O arquivo \verb$dpsmeter.jar$ não cabia na entrega do PACA. É possível baixar
ele \href{http://www.linux.ime.usp.br/~wilsonkmizutani/mac0431/ep2/dpsmeter.jar}
{\textbf{aqui}}.
 
\section{Modo de Uso}

  Para usar nosso programa, basta usar o comando do Hadoop:
  
  \begin{verbatim} $ hadoop jar dpsmeter.jar <entrada> <saída> \end{verbatim}
  
  Onde:
  
  \begin{itemize}
    \item[<entrada>]
      Deve ser um arquivo de relatório de combate gerado pelo jogo World of
      Warcraft (versão atual) como o arquivo \textbf{wowlog.txt}.
    \item[<saída>]
      É o nome do diretório onde se deseja que a saída do programa seja escrita.
      Esse diretório é criado pelo Hadoop, portanto não deve existir quando o
      programa for executado.
  \end{itemize}
  
  Temos mais três relatórios de combate bem grandes disponíveis
  \href{http://www.linux.ime.usp.br/~gorobaum/logs/}{nessa página}, caso queiram
  rodar nosso programa com uma entrada mais significativa.
  
\section{Detalhes de Implementação}

  Em geral seguimos a ideia apresentada na nossa proposta original no uso das
  funções Map e Reduce do Hadoop. As únicas grandes diferenças são:
  
  \begin{itemize}
    \item
      As estruturas "Medida", ao invés de guardarem o DPS (dano por segundo) de
      um personagem, guardam a quantidade de dano total que ele causou. Isso
      simplifica as contas na hora de juntar várias medidas nas funções Reduce e
      Shuffle.
    \item
      Nós descontamos o tempo que personagens passam mortos e inativos durante
      uma batalha, até que sejam ressucitados. Assim o tempo usado para computar
      o DPS só considera os períodos que eles realmente estiveram partipando
      ativamente da batalha.
  \end{itemize}    
  
  Para maiores detalhes de como era a ideia da proposta original, basta ler o
  arquivo \textbf{proposta.pdf} que entregamos junto com esse relatório.
  Resumidamente, temos que:
  
  \begin{enumerate}
    \item
      \textbf{Classe Map}:
      \begin{itemize}
        \item[\textit{Recebe:}] \verb$par(arquivo, texto)$
        \item[\textit{Devolve:}] \verb$par(nome, medida)$
        \item[\textit{Descrição:}]
          Lê o relatório de combate linha por linha, detectando as entradas de
          dano e ressureição dos personagens. Usa as informações de quem causou
          o dano e de quando o dano foi causado para gerar as medidas, e as de
          ressureição para descontar o tempo inativo do total nessas medidas.
      \end{itemize}
    \item
      \textbf{Classe Shuffle}:
      \begin{itemize}
        \item[\textit{Recebe:}] \verb$par(nome, lista(medidas))$
        \item[\textit{Devolve:}] \verb$par(nome, medida)$
        \item[\textit{Descrição:}]
          Junta várias medidas de um mesmo personagens numa só. Como agora as
          medidas guardam o dano total e o tempo total associado ao dano
          causado, basta somar essas informações respectivamente entre medidas.
      \end{itemize}
    \item
      \textbf{Classe Reduce}:
      \begin{itemize}
        \item[\textit{Recebe:}] \verb$par(nome, lista(medidas))$
        \item[\textit{Devolve:}] \verb$par(nome, DPS)$
        \item[\textit{Descrição:}]
          Junta várias medidas como a classe Shuffle, mas no final calcula o
          DPS de cada personagem e passa para a saída final do programa.
      \end{itemize}
  \end{enumerate}
  
  Para informações de como os relatórios do World of Warcraft são estruturados,
  recomendamos \href{http://www.wowwiki.com/API_COMBAT_LOG_EVENT}{essa página da
  Web}, que apesar de desatualizada, descreve bem a ideia básica.
  
  Com o que aprendemos nesse link e observando os relatórios gerados, fizemos a
  classe \verb$LogParser$, usada na classe Map, que interpreta as linhas do
  relatório e gera objetos do tipo \verb$LogEntry$, que abstraem entradas do
  relatório. Esses objetos sabem se a entrada que representam é de dano, de
  ressureição ou nenhuma dessas duas.
  
  A classe \verb$Map$ extrai das entradas de dano objetos do tipo \verb$Damage$
  contendo maiores detalhes sobre como o dano ocorreu. Também extrai o momento
  no tempo em que os personagens são revividos. Ele passa essas informações para
  a classe \verb$DamageMeasurer$, que mantém os dados de quanto dano cada
  personagem causou e quanto tempo cada personagem esteve efetivamente ativo no
  combate. Com isso, ela gera objetos do tipo \verb$Measure$, que são as nossas
  "medidas".
  
  Objetos do tipo \verb$Measure$ podem combinar-se, e isso é usado pelas classes
  \verb$Shuffle$ e \verb$Reduce$ para juntar as medidas e produzir o resultado
  final.
  
\section{Dificuldades e Facilidades}

  As duas maiores dificuldades que tivemos foram depurar classe \verb$LogParser$
  (principalmente a parte de ler e manipular unidades de tempo usando a classe
  \verb$Calendar$ do Java) e perceber que o Hadoop chama o método \verb$map$ de
  um mesmo objeto \verb$Map$ para cada linha do arquivo, o que exigiu cuidado na
  hora de escolher entre usar variáveis locais ou atributos de classe.
  
  Uma outra dificuldade foi que quando se instala o Hadoop usando um pacote DEB,
  a configuração padrão dele limita a memória usada para 128 MB. Depois de
  pesquisar na Internet, descobrimos que essa configuração pode ser editada no
  arquivo \textbf{hadoop-env.sh}. Mas existem dois arquivos desse: um na pasta
  onde o Hadoop é instalado, e outro em \verb$/etc/hadoop/$. Aquele que de fato
  deve ser editado é o segundo...
  
  E também tivemos a complicação de não saber como entregar o projeto pois as
  dependências não cabiam no sistema de entrega do PACA.
  
  As principais facilidades foram montar o ambiente de desenvolvimento (usando
  Eclipse e Git) e a programação em si do projeto. Ou seja, as dificuldades
  estavam mais nas partes inicial (instalação e configuração do Hadoop) e final
  (depuração em geral e entrega do projeto) do desenvolvimento.

\end{document}
