\documentclass[brazil]{beamer}
\usepackage{beamerthemesplit}
\usepackage[brazilian]{babel}
\usepackage[utf8]{inputenc}
\usepackage{color}
\usepackage{xcolor}
\usepackage{fancybox}
\usepackage{ulem}
\usepackage{upquote}
\usetheme{JuanLesPins}

\title{MAC0431 - EP2 - Proposta}
\author{Wilson Kazuo Mizutani \\ Thiago de Gouveia Nunes}

\begin{document}

\frame{\titlepage}

\section{Introdução}

\frame{
  \begin{center}
    \LARGE 1. Introdução
  \end{center}
}

\frame{
  \underline{\Large Ideia:}
  
  \pause
  \hspace{10pt}
  Usar o Hadoop para extrair estatísticas de desempenho de combate no jogo
  \textit{World of Warcraft}
  
  \vspace{20pt}
  \pause
  \underline{\Large Objetivo:} \\
  
  \pause
  \hspace{10pt}
  Permitir que o jogador meça a eficácia da formulação do seu personagem
  virtual.
  
  \vspace{20pt}
  \pause
  \underline{\Large Foco:} \\
  
  \pause
  \hspace{10pt}
  Dano por segundo (DPS) dos personagens.
}

\frame{
  \underline{\Large Base de dados:}
  
  \pause
  \vspace{10pt}
  \hspace{10pt}
  Relatórios de combate gerados automaticamente pelo jogo!
}

\section{O Problema}

\section{Abordagem}

\end{document}