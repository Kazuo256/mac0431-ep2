\documentclass[brazil]{beamer}
\usepackage{beamerthemesplit}
\usepackage[brazilian]{babel}
\usepackage[utf8]{inputenc}
\usepackage{color}
\usepackage{xcolor}
\usepackage{fancybox}
\usepackage{ulem}
\usepackage{upquote}
\usetheme{JuanLesPins}

\title{MAC0431 - EP2 - Proposta}
\author{Wilson Kazuo Mizutani \\ Thiago de Gouveia Nunes}

\begin{document}

\frame{\titlepage}

\section{1. Introdução}

\frame{
  \begin{center}
    \LARGE 1. Introdução
  \end{center}
}

\frame{
  \underline{\Large Ideia:}
  
  \pause
  \hspace{10pt}
  Usar o Hadoop para extrair estatísticas de desempenho de combate no jogo
  \textit{World of Warcraft}
  
  \vspace{20pt}
  \pause
  \underline{\Large Objetivo:} \\
  
  \pause
  \hspace{10pt}
  Permitir que o jogador meça a eficácia da formulação do seu personagem
  virtual.
  
  \vspace{20pt}
  \pause
  \underline{\Large Foco:} \\
  
  \pause
  \hspace{10pt}
  Dano por segundo (DPS) dos personagens.
}

\frame{
  \underline{\Large Base de dados:}
  
  \pause
  \vspace{10pt}
  \hspace{10pt}
  Relatórios de combate gerados automaticamente pelo jogo!
}

\section{2. O Problema}

\frame{
  \begin{center}
    \LARGE 2. O Problema
  \end{center}
}

\frame{
  \underline{\Large Entrada:}
  
  \pause
  \hspace{10pt}
  Relatório de combate gerado pelo jogo. Mais especificamente, as linhas que
  relatam danos. Elas contêm as seguintes informações:
  
  \begin{itemize}
    \pause
    \item Horário em que o dano foi causado.
    \pause
    \item Que personagem causou o dano.
    \pause
    \item Que personagem recebeu o dano.
    \pause
    \item Quantos pontos de dano foram causados.
  \end{itemize}
}

\frame{
  \underline{\Large Saída:}
  
  \pause
  \hspace{10pt}
  Uma listagem indicando para cada personagem citado na base de
  dados:
  
  \begin{itemize}
    \pause
    \item Nome do personagem.
    \pause
    \item DPS médio total do personagem.
  \end{itemize}
}

\section{3. Abordagem}

\frame{
  \begin{center}
    \LARGE 3. Abordagem
  \end{center}
}

\begin{frame}[fragile]
  \underline{\Large Função \textbf{Map}:}
  
  \vspace{4pt}
  \begin{itemize}
    \pause
    \item[Recebe] Par \verb$(Arquivo, Texto)$, como no WordCount.
    \pause
    \item[Devolve] Pares \verb$(Nome, Medida)$.
  \end{itemize}
  \pause
  \hspace{10pt}
  Onde \verb$Nome$ é o nome de um personagem e \verb$Medida$ é um par ordenado
  de dois números, \verb$(DPS, dt)$, com \verb$DPS$ sendo um DPS observado desse
  personagem no relatório e \verb$dt$ o intervalo de tempo no qual ele teve esse
  DPS.
\end{frame}

\begin{frame}[fragile]
  \underline{\Large Função \textbf{Reduce}:}
  
  \vspace{4pt}
  \begin{itemize}
    \pause
    \item[Recebe] Par \verb$(Nome, Medidas)$.
    \pause
    \item[Devolve] Par \verb$(Nome, DPS)$.
  \end{itemize}
  \pause
  \hspace{10pt}
  Onde \verb$Medidas$ é uma lista de medidas geradas pela função \textbf{Map}
  (ou agrupadas pela \textbf{Shuffle}) e \verb$DPS$ é o DPS médio total do
  personagem.
  
  \pause
  \hspace{10pt}
  O DPS médio total é na verdade a média ponderada das várias medidas de DPS
  usando as variações de tempo de cada uma delas como peso.
\end{frame}

\begin{frame}[fragile]
  \underline{\Large Função \textbf{Shuffle}:}
  
  \vspace{4pt}
  \begin{itemize}
    \pause
    \item[Recebe] Par \verb$(Nome, Medidas)$.
    \pause
    \item[Devolve] Par \verb$(Nome, Medida)$.
  \end{itemize}
  \pause
  \hspace{10pt}
  Basicamente reduz várias medidas de um mesmo personagem em uma única medida,
  usando o mesmo algoritmo que a função \textbf{Reduce} mas gerando uma saída do
  mesmo tipo que a função \textbf{Map}.
\end{frame}


\end{document}